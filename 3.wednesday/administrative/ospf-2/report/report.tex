\documentclass[14pt,a4paper]{extreport}

% Caption configuration
\usepackage{caption}

% Times New Roman, be sure to build with XeLaTeX
\usepackage{fontspec}
\usepackage[russian]{babel}
\setmainfont{Times New Roman}

% mock data
\usepackage{lipsum}

% russian
\usepackage[utf8]{inputenc}

\linespread{1.5}

\usepackage{graphicx}
\usepackage[
    left = 2cm,
    right = 1cm,
    top = 1cm,
    bottom = 2cm,
    headheight = 2cm
]{geometry}

%% New commands

% caption | screenshot index
\newcommand{\sshot}[2]{\begin{figure}[ht]%
\centering\includegraphics[width=0.8\textwidth]{../Screenshots/Screenshot_#2}%
\caption{#1}%
\label{sshot#2}%
\end{figure}%
}

\newcommand{\header}[1]{%
{
\clearpage%
\fontsize{16pt}{14pt}\selectfont
\begin{center}
\textbf{\MakeUppercase{#1}:}
\end{center}
}
}

\newcommand{\ospf}{\texttt{\MakeUppercase{ospf}}}

\newcommand{\labyear}{2017}
\newcommand{\labtitle}{Настройка сети с использованием OSPF. Индивидуальное задание по настройке OSPF}
\newcommand{\prepod}{Ананченко~И.~В.}
\newcommand{\student}{Кислюк~И.~В.}


% Configurations

% Рис 1. -> Рис 1 --, Таблица 1. -> Таблица 1 --
% Рис 1. -> Рисунок 1
\DeclareCaptionFormat{myformat}{\fontsize{12}{12}\selectfont#1#2#3}
\captionsetup[figure]{format={myformat},name={Рисунок},labelsep=endash}

\begin{document}

	\begin{titlepage}
	\begin{center}	
		\fontsize{14pt}{14pt}\selectfont
		МИНИСТЕРСТВО ОБРАЗОВАНИЯ И НАУКИ\\

		\vspace*{0.6\baselineskip}

		\MakeUppercase{Санкт-Петербургский Национальный Исследовательский Университет Информационных технологий, механики и оптики}		
		
		\vspace*{0.6\baselineskip}
		\MakeUppercase{Факультет Инфокоммуникационных технологий}\\
		\MakeUppercase{Кафедра программных систем}
	
		\vspace*{7\baselineskip}
		\fontseries{m}\fontsize{19pt}{18pt}\selectfont
		Отчет по лабораторной работе
		
		\fontseries{m}\fontsize{20pt}{18pt}\selectfont
		\textbf{\labtitle}\\
		\vspace*{1.15\baselineskip}
		\end{center}
	
	\vspace*{2\baselineskip}
	\begin{flushright}
	\fontseries{m}\fontsize{14pt}{14pt}\selectfont
	\textbf{Выполнил:}\\
	\student\\
	студент группы К4120\\
	Проверил: \prepod\\
	\end{flushright}
	
	\vspace{\fill}
	\begin{center}
	Санкт-Петербург\\
	\vspace{-1ex}
	\labyear
	\end{center}
	
\end{titlepage}

\fontsize{14pt}{14pt}\selectfont

\header{Цель работы}


Выполнить настройку дистанционно-векторного протокола маршрутизации \ospf{} на модели сети согласно варианту задания, в процессе настройки учесть, что в данной конфигурации сети присутствуют три различные \ospf -зоны

\header{Ход работы}


В первой части лабораторной работы необходимо выполнить настройку простой сети для работы с топологией \ospf .

На первом этапе аналогично предыдущей работе выполним настройку всех добавленных подсетей: соединим соответствующие интерфейсы подходящими кабелями, зададим для них IP-адреса и маски подсетей.

\sshot{Пример карскаса сети со статическими адресами}{1}

\clearpage

\sshot{Пример настройки роутера с \ospf}{2}
\sshot{Пример установки приоритета марштрута}{3}

Как видно, один из маршрутизаторов является граничным, что в целом корректно отражает действительность, т. к. именно он находится сразу в нескольких зонах.

Процесс настройки маршрутизаторов существенно не отличается от процесса, рассмотренного в предыдущей лабораторной работе, поэтому сразу приведем результат — список \ospf -соседей для указанного на рисунке маршрутизатора.

И наконец проверим, что клиентские компьютеры в разных зонах обмениваются ICMP-пакетами: выполним команду \texttt{ping}.

\sshot{Пример проверки таблицы марштрутизации роутера R2. Часть 1}{4}

От предыдущего задания структурная схема сети будет отличаться добавлением новой OSPF-зоны area 2, охватывающей следующие устройства и интерфейсы.

\sshot{Пример проверки таблицы марштрутизации роутера R2. Часть 2}{5}
\sshot{Пример проверки таблицы марштрутизации роутера R1}{6}

\clearpage

Следующим этапом было выполнение индивидуального задания согласно нарисованной схеме, приведенной на рисунке \ref{sshot0}. Шаги настройки повторяются, поэтому подробно не описываются.

\sshot{Пример индивидуального задания}{0}
\sshot{Пример каркаса сети индивидуального задания}{7}
\sshot{Пример настройки \ospf в заданной сети}{8}
\sshot{Пример настройки \texttt{ABR}-роутера в заданной сети на примере роутера R0}{9}
\sshot{Пример настройки роутера R9}{10}
\sshot{Пример таблицы маршрутизации на роутере R9}{11}
\sshot{Пример пересылки \texttt{ICMP}-пакета в сети}{12}
\sshot{Пример получения \texttt{ICMP}-пакета в сети}{13}
\sshot{Пример таблицы маршрутизации роутера R1}{14}


\header{Вывод}

В результате выполнения данной лабораторной работы была произведена настройка протокола маршрутизации \ospf {} для трех различных зон и проверена возможность обмена пакетами для каждого из клиентских компьютеров внутри них

\end{document}

