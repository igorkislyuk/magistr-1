\documentclass[14pt,a4paper]{extreport}

% Caption configuration
\usepackage{caption}

% Times New Roman, be sure to build with XeLaTeX
\usepackage{fontspec}
\usepackage[russian]{babel}
\setmainfont{Times New Roman}

% mock data
\usepackage{lipsum}

% russian
\usepackage[utf8]{inputenc}

\linespread{1.5}

\usepackage{graphicx}
\usepackage[
    letterpaper,
    left        = 2cm,
    right       = 1cm,
    top         = 1cm,
    headheight  = 2cm
]{geometry}

%% New commands

% caption | screenshot index
\newcommand{\screenshot}[2]{\begin{figure}[ht]%
\centering\includegraphics[width=0.8\textwidth]{../screenshots/screen-#2}%
\caption{#1}%
\label{picture#2}%
\end{figure}%
}

\newcommand{\header}[1]{%
{
\fontsize{16pt}{14pt}\selectfont
\begin{center}
\textbf{\MakeUppercase{#1}:}
\end{center}
}
}

\newcommand{\anan}{Ананченко~И.~В.}
\newcommand{\osipov}{Осипов~Н.~А.}
\newcommand{\igork}{Кислюк~И.~В.}
\newcommand{\egor}{Антонов~Е.~П.}
\newcommand{\igorl}{Лебедев~И.~Ю.}

\newcommand{\ping}{\texttt{ping }}

\newcommand{\dns}{\texttt{DNS}}

% Configurations

% Рис 1. -> Рис 1 --, Таблица 1. -> Таблица 1 --
% Рис 1. -> Рисунок 1
\DeclareCaptionFormat{myformat}{\fontsize{12}{12}\selectfont#1#2#3}
\captionsetup[figure]{format={myformat},name={Рисунок},labelsep=endash}

\begin{document}

	\begin{titlepage}
	\begin{center}	
		\fontsize{14pt}{14pt}\selectfont
		МИНИСТЕРСТВО ОБРАЗОВАНИЯ И НАУКИ\\

		\vspace*{0.6\baselineskip}
		
		САНКТ-ПЕТЕРБУРГСКИЙ НАЦИОНАЛЬНЫЙ ИССЛЕДОВАТЕЛЬСКИЙ УНИВЕРСИТЕТ ИНФОРМАЦИОННЫХ ТЕХНОЛОГИЙ, МЕХАНИКИ И ОПТИКИ
		
		\vspace*{0.6\baselineskip}
		ФАКУЛЬТЕТ ИНФОКОММУНИКАЦИОННЫХ ТЕХНОЛОГИЙ
		КАФЕДРА ПРОГРАММНЫХ СИСТЕМ
	
		\vspace*{7\baselineskip}
		\fontseries{m}\fontsize{19pt}{18pt}\selectfont
		Отчет по лабораторной работе
		
		\fontseries{m}\fontsize{20pt}{18pt}\selectfont
		\textbf{Прослушивание трафика с помощью ПО Wireshark}\\
		\vspace*{1.15\baselineskip}
		\end{center}
	
	\vspace*{2\baselineskip}
	\begin{flushright}
	\fontseries{m}\fontsize{14pt}{14pt}\selectfont
	\textbf{Выполнил:}\\
	\igork\\
	студент группы К4120\\
	Проверил: \anan\\
	\end{flushright}
	
	\vspace{\fill}
	\begin{center}
	Санкт-Петербург\\
	2017
	\end{center}
	
\end{titlepage}

\newpage

\header{Цель работы}

\fontsize{14pt}{14pt}\selectfont

Прослушать трафик с помощью программы WireShark, который будет сгенерирован следующими командами:

\begin{itemize}
\item Команда ping
\item Команда ping для имени сервера. Пример: \verb|ping server|
\end{itemize}

\clearpage


\header{Ход работы}

\begin{enumerate}

\item Работа проводится с двумя машинами. На первой машине установлена серверная операционная система \verb|Windows Server 2012 R2|, на второй машине установлен дистрибутив \verb|Kali Linux|. Для краткости первую машину будем именовать серверной, вторую машину -- клиентской. Узнаем \verb|ipaddress| серверной машины с помощью команды \verb|ipconfig|. Выполним команду \ping с серверной на клиентскую. Пример показан на рисунке~\ref{picture1}.

\screenshot{Пример команды \ping с серверной машины}{1}

\item Запустить программу \texttt{Wireshark} и произвести перехват запроса \ping с клиентской машины на сервеную. Пример продемонстрирован на рисунках~\ref{picture2} и \ref{picture3}.

\screenshot{Пример перехвата запроса при помощи программы Wireshark}{2}

\screenshot{Пример детального вида для команды \ping}{3}

\item Необходимо выполнить настройку \dns сервера для выполнения запроса на определение ip-адреса через имя сервера. Этапы следующие.

\item Настройка определенных ролей на серверой машине. Сначала выбираются необходимые роли и надстройки над ними. Примеры показаны на рисунках \ref{picture4}, \ref{picture5}, \ref{picture6}.

\screenshot{Выбор роли \dns {} среди доступных ролей}{4}
\screenshot{Подтверждение установки \dns {} ролей}{5}
\screenshot{Завершение установки \dns {} ролей и надстроек}{6}

\item Следующий этапом стала установка зоны и ручное добавление соотвествие имени (в данном случае \texttt{server} к локальному адресу сервеной машины). Процесс добавления зоны продемонстрирован на рисунках~\ref{picture7}, \ref{picture8}, \ref{picture9}, \ref{picture10}.

\screenshot{Создание зоны типа \texttt{forward lookup zone}}{7}
\screenshot{Установка имени зоны}{8}
\screenshot{Выбор файла для хранение значений в данной зоне}{9}
\screenshot{Завершение настройки зоны}{10}

\item Далее необходимо занести соответствие локального адреса и имени компьютера в зону.

\item После предыдущего этапа возможно выполнить запрос на разрещение адреса компьютера через имя в таблице \dns . Примеры показаны на рисунках~\ref{picture11}, \ref{picture12}.

\screenshot{Пример возможного списка пакетов при выполнении запроса \dns}{11}
\screenshot{Пример окна терминала, в котором происходит разрешение имени \dns}{12}

\end{enumerate}

\clearpage

\header{Вывод}

В ходе лабораторной работы была построена сеть с двумя виртуальными машинами, на одной из которых был установлен 
дистрибутив \texttt{Kali Linux}. На прослушиваемой машине была поднята роль \dns {}-сервера. \\

При помощи входящей в дистрибутив программы \texttt{WireShark} были перехвачены следующие типы запросов:

\begin{itemize}
\item \ping запрос
\item \dns {} запрос 
\end{itemize}

В результате были получены навыки использования специального ПО для перехвата и просмотра трафика в сети.

\end{document}

