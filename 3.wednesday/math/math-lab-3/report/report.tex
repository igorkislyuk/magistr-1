\documentclass[14pt,a4paper]{extreport}

% Caption configuration
\usepackage{caption}

% Times New Roman, be sure to build with XeLaTeX
\usepackage{fontspec}
\usepackage[russian]{babel}
\setmainfont{Times New Roman}

% mock data
\usepackage{lipsum}

% russian
\usepackage[utf8]{inputenc}

\linespread{1.5}

\usepackage{graphicx}
\usepackage[
    left = 2cm,
    right = 1cm,
    top = 1cm,
    bottom = 2cm,
    headheight = 2cm
]{geometry}

%% New commands

% caption | screenshot index
\newcommand{\sshot}[2]{\begin{figure}[ht]%
\centering\includegraphics[width=0.8\textwidth]{../Screenshots/Screenshot_#2}%
\caption{#1}%
\label{sshot#2}%
\end{figure}%
}

\newcommand{\header}[1]{%
{
\clearpage%
\fontsize{16pt}{14pt}\selectfont
\begin{center}
\textbf{\MakeUppercase{#1}:}
\end{center}
}
}

\newcommand{\labyear}{2017}
\newcommand{\labtitle}{Корреляционный анализ}
\newcommand{\prepod}{Ананченко~И.~В.}
\newcommand{\student}{Кислюк~И.~В.}


% Configurations

% Рис 1. -> Рис 1 --, Таблица 1. -> Таблица 1 --
% Рис 1. -> Рисунок 1
\DeclareCaptionFormat{myformat}{\fontsize{12}{12}\selectfont#1#2#3}
\captionsetup[figure]{format={myformat},name={Рисунок},labelsep=endash}

\begin{document}

	\begin{titlepage}
	\begin{center}	
		\fontsize{14pt}{14pt}\selectfont
		МИНИСТЕРСТВО ОБРАЗОВАНИЯ И НАУКИ\\

		\vspace*{0.6\baselineskip}

		\MakeUppercase{Санкт-Петербургский Национальный Исследовательский Университет Информационных технологий, механики и оптики}		
		
		\vspace*{0.6\baselineskip}
		\MakeUppercase{Факультет Инфокоммуникационных технологий}\\
		\MakeUppercase{Кафедра программных систем}
	
		\vspace*{7\baselineskip}
		\fontseries{m}\fontsize{19pt}{18pt}\selectfont
		Отчет по лабораторной работе
		
		\fontseries{m}\fontsize{20pt}{18pt}\selectfont
		\textbf{\labtitle}\\
		\vspace*{1.15\baselineskip}
		\end{center}
	
	\vspace*{2\baselineskip}
	\begin{flushright}
	\fontseries{m}\fontsize{14pt}{14pt}\selectfont
	\textbf{Выполнил:}\\
	\student\\
	студент группы К4120\\
	Проверил: \prepod\\
	\end{flushright}
	
	\vspace{\fill}
	\begin{center}
	Санкт-Петербург\\
	\vspace{-1ex}
	\labyear
	\end{center}
	
\end{titlepage}

\fontsize{14pt}{14pt}\selectfont

\header{Цель работы}


Изучить вопрос о нахождении автокорреляционной функции.

\header{Ход работы}


\textbf{Пример 4.1. } Приведем пример вычисления корреляционных характеристик 
случайных сигналов.
 
\sshot{Листинг вычисления автокорреляционной функции и коэффициента корреляции сигнала с нормальным распределением вероятностей}{1}

\clearpage

\textbf{Пример 4.2. } Вычислим функцию и коэффициент взаимной корреляции для 
двух сигналов, заданных в виде функций X1 – с нормальным распределением 
вероятностей и параметрами $E = 1$ – математическое ожидание, $\sigma = 0,5$ – 
стандартное отклонение, $K \times 1023$ и $X2$ – с $\beta$ -распределением вероятностей и параметрами $u = 20$, $v = 4$.
 
\sshot{Листинг вычисления функции и коэффициента взаимной корреляции}{2}

\clearpage

\textbf{Пример 4.3. } Вычислим автокорреляционную функцию сигнала $Rn$, заданного функцией Вейерштрасса.

\sshot{Фрагмент автокорреляционной функции заданного сигнала}{3}

\clearpage

\sshot{Автокорреляционная функция и $RCn$ коэффициент корреляции для сигнала}{4}
 
\textbf{Упражнение 2. } Вычислите автокорреляционную функцию и коэффициент корреляции для сигнала, заданного в виде функции с $X1$ - распределением вероятностей и 
параметрами $u = 18$, $v = 5$, $k = 0,1..1023$. Постройте график автокорреляционной функции в зависимости от числа $n$, где $n = 0,1..100$.

\clearpage
 
\sshot{Автокорреляционная функция и коэффициент корреляции}{5}
\sshot{Коэффициент корреляции}{6} 

\header{Вывод}

В ходе лабораторной работы был изучен способ нахождения 
коэффициента корреляции и процесс построения автокорреляционной 
функции.

\end{document}

