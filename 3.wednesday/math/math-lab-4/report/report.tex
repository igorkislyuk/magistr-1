\documentclass[14pt,a4paper]{extreport}

% Caption configuration
\usepackage{caption}

% Times New Roman, be sure to build with XeLaTeX
\usepackage{fontspec}
\usepackage[russian]{babel}
\setmainfont{Times New Roman}

% mock data
\usepackage{lipsum}

% russian
\usepackage[utf8]{inputenc}

\linespread{1.5}

\usepackage{graphicx}
\usepackage[
    left = 2cm,
    right = 1cm,
    top = 1cm,
    bottom = 2cm,
    headheight = 2cm
]{geometry}

%% New commands

% caption | screenshot index
\newcommand{\sshot}[2]{\begin{figure}[ht]%
\centering\includegraphics[width=0.8\textwidth]{../Screenshots/Screenshot_#2}%
\caption{#1}%
\label{sshot#2}%
\end{figure}%
}

\newcommand{\header}[1]{%
{
\clearpage%
\fontsize{16pt}{14pt}\selectfont
\begin{center}
\textbf{\MakeUppercase{#1}:}
\end{center}
}
}

\newcommand{\labyear}{2017}
\newcommand{\labtitle}{Спектральный анализ}
\newcommand{\prepod}{Ананченко~И.~В.}
\newcommand{\student}{Кислюк~И.~В.}

\newcommand{\mathcad}{\texttt{Mathcad}}
\newcommand{\matlab}{\texttt{Matlab}}

% Configurations

% Рис 1. -> Рис 1 --, Таблица 1. -> Таблица 1 --
% Рис 1. -> Рисунок 1
\DeclareCaptionFormat{myformat}{\fontsize{12}{12}\selectfont#1#2#3}
\captionsetup[figure]{format={myformat},name={Рисунок},labelsep=endash}

\begin{document}

	\begin{titlepage}
	\begin{center}	
		\fontsize{14pt}{14pt}\selectfont
		МИНИСТЕРСТВО ОБРАЗОВАНИЯ И НАУКИ\\

		\vspace*{0.6\baselineskip}

		\MakeUppercase{Санкт-Петербургский Национальный Исследовательский Университет Информационных технологий, механики и оптики}		
		
		\vspace*{0.6\baselineskip}
		\MakeUppercase{Факультет Инфокоммуникационных технологий}\\
		\MakeUppercase{Кафедра программных систем}
	
		\vspace*{7\baselineskip}
		\fontseries{m}\fontsize{19pt}{18pt}\selectfont
		Отчет по лабораторной работе
		
		\fontseries{m}\fontsize{20pt}{18pt}\selectfont
		\textbf{\labtitle}\\
		\vspace*{1.15\baselineskip}
		\end{center}
	
	\vspace*{2\baselineskip}
	\begin{flushright}
	\fontseries{m}\fontsize{14pt}{14pt}\selectfont
	\textbf{Выполнил:}\\
	\student\\
	студент группы К4120\\
	Проверил: \prepod\\
	\end{flushright}
	
	\vspace{\fill}
	\begin{center}
	Санкт-Петербург\\
	\vspace{-1ex}
	\labyear
	\end{center}
	
\end{titlepage}

\fontsize{14pt}{14pt}\selectfont

\header{Цель работы}

Изучить спектральный анализ, Фурье-преобразование сигнала, спектральный анализ случайных сигналов и выполнить примеры в \mathcad{} и \matlab{}.

\header{Ход работы}

Выполним примеры реализации преобразований Фурье в пакете \mathcad{} и \matlab{}

\vspace{50pt}

\textbf{Пример 5.1 } Прямое и обратное преобразование Фурье детерминированного сигнала.

\sshot{Пример прямого и обратного преобразования Фурье детерминированного сигнала.}{1}

\clearpage

\textbf{Пример 5.2 } Прямое и обратное преобразование Фурье случайного сигнала.

\sshot{Прямое и обратное преобразование Фурье случайного сигнала}{2}

\clearpage

\textbf{Пример 5.3 } Прямое и обратное преобразования Фурье в пакете \matlab {}. Приведем пример реализации быстрого преобразования Фурье в \matlab{} для дискретной функции.

\sshot{Графики сигнала и его прямого и обратного преобразований Фурье}{3}

\header{Вывод}


В ходе практической работы, был изучен спектральный анализ, Фурье-преобразование случайного сигнала и детерминированного сигнала. Также были сделаны примеры в \mathcad{} и \matlab{}, с помощью которых можно построить графики сигналов.

\end{document}

