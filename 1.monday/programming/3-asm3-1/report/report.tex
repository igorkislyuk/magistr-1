\documentclass[14pt,a4paper]{extreport}

% Caption configuration
\usepackage{caption}

% Times New Roman, be sure to build with XeLaTeX
\usepackage{fontspec}
\usepackage[russian]{babel}
\setmainfont{Times New Roman}

% mock data
\usepackage{lipsum}

% russian
\usepackage[utf8]{inputenc}

\linespread{1.5}

\usepackage{graphicx}
\usepackage[
    left = 2cm,
    right = 1cm,
    top = 1cm,
    bottom = 2cm,
    headheight = 2cm
]{geometry}

%% New commands

% caption | screenshot index
\newcommand{\sshot}[2]{\begin{figure}[ht]%
\centering\includegraphics[width=0.8\textwidth]{../Screenshots/Screenshot_#2}%
\caption{#1}%
\label{sshot#2}%
\end{figure}%
}

\newcommand{\header}[1]{%
{
\clearpage%
\fontsize{16pt}{14pt}\selectfont
\begin{center}
\textbf{\MakeUppercase{#1}:}
\end{center}
}
}

\newcommand{\labyear}{2017}
\newcommand{\labtitle}{Основы программирования микроконтроллеров}
\newcommand{\prepod}{Осипов~Н.~А.}
\newcommand{\student}{Кислюк~И.~В.}


% Configurations

% Рис 1. -> Рис 1 --, Таблица 1. -> Таблица 1 --
% Рис 1. -> Рисунок 1
\DeclareCaptionFormat{myformat}{\fontsize{12}{12}\selectfont#1#2#3}
\captionsetup[figure]{format={myformat},name={Рисунок},labelsep=endash}

\begin{document}

	\begin{titlepage}
	\begin{center}	
		\fontsize{14pt}{14pt}\selectfont
		МИНИСТЕРСТВО ОБРАЗОВАНИЯ И НАУКИ\\

		\vspace*{0.6\baselineskip}

		\MakeUppercase{Санкт-Петербургский Национальный Исследовательский Университет Информационных технологий, механики и оптики}		
		
		\vspace*{0.6\baselineskip}
		\MakeUppercase{Факультет Инфокоммуникационных технологий}\\
		\MakeUppercase{Кафедра программных систем}
	
		\vspace*{7\baselineskip}
		\fontseries{m}\fontsize{19pt}{18pt}\selectfont
		Отчет по лабораторной работе
		
		\fontseries{m}\fontsize{20pt}{18pt}\selectfont
		\textbf{\labtitle}\\
		\vspace*{1.15\baselineskip}
		\end{center}
	
	\vspace*{2\baselineskip}
	\begin{flushright}
	\fontseries{m}\fontsize{14pt}{14pt}\selectfont
	\textbf{Выполнил:}\\
	\student\\
	студент группы К4120\\
	Проверил: \prepod\\
	\end{flushright}
	
	\vspace{\fill}
	\begin{center}
	Санкт-Петербург\\
	\vspace{-1ex}
	\labyear
	\end{center}
	
\end{titlepage}

\fontsize{14pt}{14pt}\selectfont

\header{Цель работы}


\begin{itemize}

\item изучение назначения и особенностей архитектуры однокристальных микроконтроллеров;

\item ознакомление с архитектурой и программной моделью AVR-микроконтроллеров;

\item изучение этапов разработки ПО для встраиваемых микропроцессоров;

\item приобретение навыков работы в среде AVR Studio.

\end{itemize}

\header{Ход работы}


\begin{itemize}

\item Создадим программу по образцу, при помощи AVR Studio и проверим по шагам (строчкам) её работу. Для этого расставим точки останова на необходимых строчках и пошагово пройдём по ним. В процессе отладки программы можно заметить, как поочерёдно заполняются регистры соответствующие работающими с ними командами

\sshot{Пример листинга первой программы}{3}

\item Создадим программу вычисления произведения и суммы двух чисел А и В, находящихся в РОН. Затем из суммы А и В вычесть число С. Числа А, В и С выбираются в соответствии с номером варианта. В начало программы нужно поместить директиву «.device» для микроконтроллера \texttt{ATmega8535}. В комментариях указаны фамилия и номер группы. Пример состояний регистров во время выполнения программы показаны на рисунка \ref{sshot4} -- \ref{sshot6}

\sshot{Пример программы с индиивидуальным заданием. Начальный запуск}{4}
\sshot{Пример программы с индиивидуальным заданием. Регистры после выполнения умножения}{5}
\sshot{Пример программы с индиивидуальным заданием. Регистры после проведение операции вычитания}{6}

\item В процессе отладки можно увидеть, как происходит изменение значений регистра. Также можно заметить, что при умножении удваивается количество разрядов, что приводит к занятию результатом умножения двух РОН.

\end{itemize}

\header{Вывод}


\textbf{Сигналы состояния} характеризуют текущие параметры объекта управления. Они формируются путём преобразования выходных сигналов датчиков (Д) с помощью \\ аналого--цифровых преобразователей (АЦП) или формирователей сигналов состояния (ФСС); последние чаще всего выполняют функции гальванической развязки и формирования уровней.

\par

\textbf{Сигналы управления}, выработанные микроконтроллером, подвергаются преобразованию с помощью цифро-аналоговых преобразователей (ЦАП) или формирователей сигналов управления (ФСУ), в качестве которых применяются усилители мощности, оптроны, транзисторные. Выходные сигналы ЦАП и ФСУ представляют собой соответственно аналоговые и дискретные управляющие воздействия, которые поступают на исполнительные устройства (ИУ). В системе могут быть также предусмотрены панель управления, устройство индикации и интерфейс для обмена информацией с внешними устройствами. В зависимости от назначения и характеристик конкретной системы некоторые из указанных элементов могут отсутствовать.

\end{document}

