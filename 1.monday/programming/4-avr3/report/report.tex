\documentclass[14pt,a4paper]{extreport}

% Caption configuration
\usepackage{caption}

% Times New Roman, be sure to build with XeLaTeX
\usepackage{fontspec}
\usepackage[russian]{babel}
\setmainfont{Times New Roman}

% mock data
\usepackage{lipsum}

% russian
\usepackage[utf8]{inputenc}

\linespread{1.5}

\usepackage{graphicx}
\usepackage[
    left = 2cm,
    right = 1cm,
    top = 1cm,
    bottom = 2cm,
    headheight = 2cm
]{geometry}

%% New commands

% caption | screenshot index
\newcommand{\sshot}[2]{\begin{figure}[ht]%
\centering\includegraphics[width=0.8\textwidth]{../Screenshots/#2}%
\caption{#1}%
\label{sshot#2}%
\end{figure}%
}

\newcommand{\header}[1]{%
{
\clearpage%
\fontsize{16pt}{14pt}\selectfont
\begin{center}
\textbf{\MakeUppercase{#1}:}
\end{center}
}
}

\newcommand{\labyear}{2017}
\newcommand{\labtitle}{Разработка программы в среде CODEVISION AVR}
\newcommand{\prepod}{Осипов~Н.~А.}
\newcommand{\student}{Кислюк~И.~В.}


% Configurations

% Рис 1. -> Рис 1 --, Таблица 1. -> Таблица 1 --
% Рис 1. -> Рисунок 1
\DeclareCaptionFormat{myformat}{\fontsize{12}{12}\selectfont#1#2#3}
\captionsetup[figure]{format={myformat},name={Рисунок},labelsep=endash}

\begin{document}

	\begin{titlepage}
	\begin{center}	
		\fontsize{14pt}{14pt}\selectfont
		МИНИСТЕРСТВО ОБРАЗОВАНИЯ И НАУКИ\\

		\vspace*{0.6\baselineskip}

		\MakeUppercase{Санкт-Петербургский Национальный Исследовательский Университет Информационных технологий, механики и оптики}		
		
		\vspace*{0.6\baselineskip}
		\MakeUppercase{Факультет Инфокоммуникационных технологий}\\
		\MakeUppercase{Кафедра программных систем}
	
		\vspace*{7\baselineskip}
		\fontseries{m}\fontsize{19pt}{18pt}\selectfont
		Отчет по лабораторной работе
		
		\fontseries{m}\fontsize{20pt}{18pt}\selectfont
		\textbf{\labtitle}\\
		\vspace*{1.15\baselineskip}
		\end{center}
	
	\vspace*{2\baselineskip}
	\begin{flushright}
	\fontseries{m}\fontsize{14pt}{14pt}\selectfont
	\textbf{Выполнил:}\\
	\student\\
	студент группы К4120\\
	Проверил: \prepod\\
	\end{flushright}
	
	\vspace{\fill}
	\begin{center}
	Санкт-Петербург\\
	\vspace{-1ex}
	\labyear
	\end{center}
	
\end{titlepage}

\fontsize{14pt}{14pt}\selectfont

\header{Цель работы}


\begin{enumerate}

\item Изучение среды программирования CodeVision AVR;
\item Приобретение навыков программирования аппаратных средств.

\end{enumerate}

\header{Ход работы}

\begin{itemize}

\item Изучим среду разработки и разработаем микроконтроллерное устройство, реализующее простую функцию – мигание светодиода. Для этого создадим программу по шаблону, используя следующие параметры: тип процессора – AT90, тип микроконтроллера – ATmega8, рабочая частота – 8 МГц

\sshot{Полученный шаблон программы}{1}

\item Заменим шаблон программы требуемым кодом и скомпилируем программу. 

\sshot{Компиляция программы}{2}

\clearpage

\item Выполним отладку программы. Для этого откроем файл .coff при помощи AVR Studio 4. Сохраним предложенный проект отладки

\sshot{Проект отладки в среде разработки AVR Studio 4}{3}

\item Выполним отладку. Для этого установим точку останова в начале метода main() и пошагово начнём выполнять программу. Заметим, что при попадании в цикл while(1) программа «войдёт в петлю». Таким образом, мы реализовали непрерывное мигание светодиода с частотой 10Гц.

\sshot{Отладка программы}{4}

\end{itemize}

\header{Вывод}

	В результате выполнения лабораторной работы были получены навыки разработки программного обеспечения для встраиваемых микропроцессоров. Также был получен опыт работы в среде AVR Studio.

\end{document}

