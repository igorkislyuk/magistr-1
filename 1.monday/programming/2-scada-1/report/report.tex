\documentclass[14pt,a4paper]{extreport}

% Caption configuration
\usepackage{caption}

% Times New Roman, be sure to build with XeLaTeX
\usepackage{fontspec}
\usepackage[russian]{babel}
\setmainfont{Times New Roman}

% mock data
\usepackage{lipsum}

% russian
\usepackage[utf8]{inputenc}

\linespread{1.5}

\usepackage{graphicx}
\usepackage[
    left = 2cm,
    right = 1cm,
    top = 1cm,
    bottom = 2cm,
    headheight = 2cm
]{geometry}

%% New commands

% caption | screenshot index
\newcommand{\sshot}[2]{\begin{figure}[ht]%
\centering\includegraphics[width=0.8\textwidth]{../Screenshots/Screenshot_#2}%
\caption{#1}%
\label{sshot#2}%
\end{figure}%
}

\newcommand{\header}[1]{%
{
\clearpage%
\fontsize{16pt}{14pt}\selectfont
\begin{center}
\textbf{\MakeUppercase{#1}:}
\end{center}
}
}

\newcommand{\labyear}{2017}
\newcommand{\labtitle}{Создание простейшего проекта}
\newcommand{\prepod}{Осипов~Н.~А.}
\newcommand{\student}{Кислюк~И.~В.}


% Configurations

% Рис 1. -> Рис 1 --, Таблица 1. -> Таблица 1 --
% Рис 1. -> Рисунок 1
\DeclareCaptionFormat{myformat}{\fontsize{12}{12}\selectfont#1#2#3}
\captionsetup[figure]{format={myformat},name={Рисунок},labelsep=endash}

\begin{document}

	\begin{titlepage}
	\begin{center}	
		\fontsize{14pt}{14pt}\selectfont
		МИНИСТЕРСТВО ОБРАЗОВАНИЯ И НАУКИ\\

		\vspace*{0.6\baselineskip}

		\MakeUppercase{Санкт-Петербургский Национальный Исследовательский Университет Информационных технологий, механики и оптики}		
		
		\vspace*{0.6\baselineskip}
		\MakeUppercase{Факультет Инфокоммуникационных технологий}\\
		\MakeUppercase{Кафедра программных систем}
	
		\vspace*{7\baselineskip}
		\fontseries{m}\fontsize{19pt}{18pt}\selectfont
		Отчет по лабораторной работе
		
		\fontseries{m}\fontsize{20pt}{18pt}\selectfont
		\textbf{\labtitle}\\
		\vspace*{1.15\baselineskip}
		\end{center}
	
	\vspace*{2\baselineskip}
	\begin{flushright}
	\fontseries{m}\fontsize{14pt}{14pt}\selectfont
	\textbf{Выполнил:}\\
	\student\\
	студент группы К4120\\
	Проверил: \prepod\\
	\end{flushright}
	
	\vspace{\fill}
	\begin{center}
	Санкт-Петербург\\
	\vspace{-1ex}
	\labyear
	\end{center}
	
\end{titlepage}

\fontsize{14pt}{14pt}\selectfont

\header{Цель работы}


Освоить методику создания системы мониторинга, содержащую один узел АРМ с использованием механизма автопостроения каналов TRACE MODE методом \textbf{от шаблона экрана}.

\header{Ход работы}


Работа выполняется согласно пунктам задания. Примеры выполнения заданий приведены на соотвествующих рисунках.

\begin{enumerate}

\item Создание нового проекта. Пример приведен на рисунке \ref{sshot1}

\sshot{Пример окна создания нового проекта}{1}

\item Создание графического текста. Пример приведен на рисунке \ref{sshot2}

\sshot{Пример создания текста}{2}

\item Создание динамического текста, создание аргумента экрана в процессе настройки текста. Пример на рисунке \ref{sshot3}

\sshot{Пример добавления привязки}{3}

\item Выполним автопостроение канала и созданим генератор синуса, пример приведен на рисунке \ref{sshot4}

\sshot{Пример создания генератора синуса}{4}

\item Выполним тестовый запуск проекта проекта. Пример показан на рисунке  \ref{sshot5}

\sshot{Пример тестового запуска проекта}{5}

\item Теперь выполним добавление функции управления. Пример показан на рисунке~\ref{sshot6}

\sshot{Пример добавления функции Control}{6}

\item Для отображения добавим ГЭ тренд. Пример его работы показан на рисунке \ref{sshot7}

\clearpage

\sshot{Пример работы элемента}{7}

\item Создадим аргументы и выберем язык ST. Примеры показаны на рисунках \ref{sshot8} и \ref{sshot9}

\sshot{Пример использования языка ST}{8}

\sshot{Пример программы на языке ST}{9}

\clearpage

\item Необходимо выполнить привязку аттрибутов, а также произвести тестовые запуски в TraceMode, и выполнить эксперимент с выводом данных в Excel. Примеры показаны на рисунках \ref{sshot10} -- \ref{sshot12}

\sshot{Пример связывания данных}{10}
\sshot{Пример финального тестового запуска}{11}
\sshot{Пример вывода данных в Excel}{12}

\end{enumerate}

\header{Вывод}


В результате проделанной работы был построен экспериментальный стенд для суммирования значений.

\end{document}

