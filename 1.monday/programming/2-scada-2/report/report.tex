\documentclass[14pt,a4paper]{extreport}

% Caption configuration
\usepackage{caption}

% Times New Roman, be sure to build with XeLaTeX
\usepackage{fontspec}
\usepackage[russian]{babel}
\setmainfont{Times New Roman}

% mock data
\usepackage{lipsum}

% russian
\usepackage[utf8]{inputenc}

\linespread{1.5}

\usepackage{graphicx}
\usepackage[
    left = 2cm,
    right = 1cm,
    top = 1cm,
    bottom = 2cm,
    headheight = 2cm
]{geometry}

%% New commands

% caption | screenshot index
\newcommand{\sshot}[2]{\begin{figure}[ht]%
\centering\includegraphics[width=0.8\textwidth]{../Screenshots/Screenshot_#2}%
\caption{#1}%
\label{sshot#2}%
\end{figure}%
}

\newcommand{\header}[1]{%
{
\clearpage%
\fontsize{16pt}{14pt}\selectfont
\begin{center}
\textbf{\MakeUppercase{#1}:}
\end{center}
}
}

\newcommand{\labyear}{2017}
\newcommand{\labtitle}{Реализация логических функций на языке Техно FBD}
\newcommand{\prepod}{Осипов~Н.~А.}
\newcommand{\student}{Кислюк~И.~В.}


% Configurations

% Рис 1. -> Рис 1 --, Таблица 1. -> Таблица 1 --
% Рис 1. -> Рисунок 1
\DeclareCaptionFormat{myformat}{\fontsize{12}{12}\selectfont#1#2#3}
\captionsetup[figure]{format={myformat},name={Рисунок},labelsep=endash}

\begin{document}

	\begin{titlepage}
	\begin{center}	
		\fontsize{14pt}{14pt}\selectfont
		МИНИСТЕРСТВО ОБРАЗОВАНИЯ И НАУКИ\\

		\vspace*{0.6\baselineskip}

		\MakeUppercase{Санкт-Петербургский Национальный Исследовательский Университет Информационных технологий, механики и оптики}		
		
		\vspace*{0.6\baselineskip}
		\MakeUppercase{Факультет Инфокоммуникационных технологий}\\
		\MakeUppercase{Кафедра программных систем}
	
		\vspace*{7\baselineskip}
		\fontseries{m}\fontsize{19pt}{18pt}\selectfont
		Отчет по лабораторной работе
		
		\fontseries{m}\fontsize{20pt}{18pt}\selectfont
		\textbf{\labtitle}\\
		\vspace*{1.15\baselineskip}
		\end{center}
	
	\vspace*{2\baselineskip}
	\begin{flushright}
	\fontseries{m}\fontsize{14pt}{14pt}\selectfont
	\textbf{Выполнил:}\\
	\student\\
	студент группы К4120\\
	Проверил: \prepod\\
	\end{flushright}
	
	\vspace{\fill}
	\begin{center}
	Санкт-Петербург\\
	\vspace{-1ex}
	\labyear
	\end{center}
	
\end{titlepage}

\fontsize{14pt}{14pt}\selectfont

\header{Цель работы}


Необходимо освоить методику программирования логических функций при помощи SCADA--системы TRACE MODE на языке Техно FBD.

\header{Ход работы}


\begin{itemize}

\item Разработка проекта начинается с запуска интегрированной среды разработки. Создадим новый проект, и добавим на него указанные элементы текстовых полей для визуализации ввода и вывода данных. Пример показан на рисунке \ref{sshot1}

\sshot{Пример создания графических элементов}{1}

\item Добавим на экран кнопки для ввода данных и создадим новые аргументы при помощи привязки. 

\sshot{Пример создания аргументов}{2}

\clearpage

\item Создадим программу (индивидуальное задание) на языке Техно FBD, используя логические элементы. На основе индивидуального задания номер 11, создадим новую логическую функцию при помощи SCADA–системы TRACE MODE на языке Техно FBD. Пример решенной задачи приведен на рисунке \ref{sshot3}

\sshot{Пример выполнения контрольного задания на языке Техно FBD}{3}

\clearpage

\item Проверим работоспособность созданной программы визуально. Примеры представлены на рисунках \ref{sshot4} -- \ref{sshot5}.

\sshot{Пример визуальной проверки рабоспособной программы}{4}

\sshot{Пример проверки работоспособности программы в другой IDE}{5}

\end{itemize}

\header{Вывод}


В результате выполнения лабораторной работы были получены навыки построения логических функций при помощи SCADA--системы TRACE MODE на языке Техно FBD.

\end{document}

