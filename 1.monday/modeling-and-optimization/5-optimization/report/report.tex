\documentclass[14pt,a4paper]{extreport}

% Caption configuration
\usepackage{caption}

% Times New Roman, be sure to build with XeLaTeX
\usepackage{fontspec}
\usepackage[russian]{babel}
\setmainfont{Times New Roman}

% mock data
\usepackage{lipsum}

% russian
\usepackage[utf8]{inputenc}

\linespread{1.5}

\usepackage{graphicx}
\usepackage[
    left = 2cm,
    right = 1cm,
    top = 1cm,
    bottom = 2cm,
    headheight = 2cm
]{geometry}

%% New commands

% caption | screenshot index
\newcommand{\sshot}[2]{\begin{figure}[ht]%
\centering\includegraphics[width=0.8\textwidth]{../Screenshots/Screenshot_#2}%
\caption{#1}%
\label{sshot#2}%
\end{figure}%
}

\newcommand{\header}[1]{%
{
\clearpage%
\fontsize{16pt}{14pt}\selectfont
\begin{center}
\textbf{\MakeUppercase{#1}:}
\end{center}
}
}

\newcommand{\matlab}{\texttt{\MakeUppercase{matlab}}}
\newcommand{\labyear}{2017}
\newcommand{\labtitle}{Методы оптимизации в \MakeUppercase{Matlab}. Задачи линейного программирования}
\newcommand{\prepod}{Осипов~Н.~А.}
\newcommand{\student}{Кислюк~И.~В.}


% Configurations

% Рис 1. -> Рис 1 --, Таблица 1. -> Таблица 1 --
% Рис 1. -> Рисунок 1
\DeclareCaptionFormat{myformat}{\fontsize{12}{12}\selectfont#1#2#3}
\captionsetup[figure]{format={myformat},name={Рисунок},labelsep=endash}

\begin{document}

	\begin{titlepage}
	\begin{center}	
		\fontsize{14pt}{14pt}\selectfont
		МИНИСТЕРСТВО ОБРАЗОВАНИЯ И НАУКИ\\

		\vspace*{0.6\baselineskip}

		\MakeUppercase{Санкт-Петербургский Национальный Исследовательский Университет Информационных технологий, механики и оптики}		
		
		\vspace*{0.6\baselineskip}
		\MakeUppercase{Факультет Инфокоммуникационных технологий}\\
		\MakeUppercase{Кафедра программных систем}
	
		\vspace*{7\baselineskip}
		\fontseries{m}\fontsize{19pt}{18pt}\selectfont
		Отчет по лабораторной работе
		
		\fontseries{m}\fontsize{20pt}{18pt}\selectfont
		\textbf{\labtitle}\\
		\vspace*{1.15\baselineskip}
		\end{center}
	
	\vspace*{2\baselineskip}
	\begin{flushright}
	\fontseries{m}\fontsize{14pt}{14pt}\selectfont
	\textbf{Выполнил:}\\
	\student\\
	студент группы К4120\\
	Проверил: \prepod\\
	\end{flushright}
	
	\vspace{\fill}
	\begin{center}
	Санкт-Петербург\\
	\vspace{-1ex}
	\labyear
	\end{center}
	
\end{titlepage}

\fontsize{14pt}{14pt}\selectfont

\header{Цель работы}

Изучить и использовать методы оптимизации в \matlab.

\header{Ход работы}

\begin{enumerate}

\item Рассмотрим задачи безусловной оптимизации на примере градиентных методов нахождения экстремума (рисунки \ref{sshot1} - \ref{sshot3})

\sshot{Метод Франка-Вульфа}{1}
\sshot{Метод штрафных функций}{2}
\sshot{Метод Эрроу-Гурвица}{3}

\clearpage

\item Далее, решим задачу (указана на рисунке \ref{sshot6}) линейного программирования с помощью функции \texttt{linprog}.

\sshot{Задача линейного программирования}{6}

\textbf{Требуется:}

\begin{itemize}

\item Найти такой план производства продукции (количество единиц каждого вида продукции), при котором прибыль от ее реализации будет максимальной при условии, что потребление ресурсов по каждому виду продукции не превзойдет имеющихся запасов.

\item Найти такой набор ресурсов, при котором общие затраты на ресурсы будут минимальными при условии, что затраты на ресурсы при производстве каждого вида продукции будут не менее прибыли от реализации этой продукции 

\end{itemize}

\item Сформируем данные условия в виде системы уравнений и функции, максимум (или минимум) которой необходимо вычислить, и в матричном виде передадим эти данные в функцию \texttt{linprog}.

На рисунках \ref{sshot4}, \ref{sshot5} представлены решения составленных систем, отвечающие условиям задач.

\sshot{Решение задачи линейного программирования. Пункт 1}{4}
\sshot{Решение задачи линейного программирования. Пункт 2}{5}

\end{enumerate}

\header{Вывод}

Успешно исследовали и применили на практике методы оптимизации в \matlab .

\end{document}

