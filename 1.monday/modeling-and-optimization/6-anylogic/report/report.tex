\documentclass[14pt,a4paper]{extreport}

% Caption configuration
\usepackage{caption}

% Times New Roman, be sure to build with XeLaTeX
\usepackage{fontspec}
\usepackage[russian]{babel}
\setmainfont{Times New Roman}

% mock data
\usepackage{lipsum}

% russian
\usepackage[utf8]{inputenc}

\linespread{1.5}

\usepackage{graphicx}
\usepackage[
    left = 2cm,
    right = 1cm,
    top = 1cm,
    bottom = 2cm,
    headheight = 2cm
]{geometry}

%% New commands

% caption | screenshot index
\newcommand{\sshot}[2]{\begin{figure}[ht]%
\centering\includegraphics[width=0.8\textwidth]{../Screenshots/Screenshot_#2}%
\caption{#1}%
\label{sshot#2}%
\end{figure}%
}

\newcommand{\header}[1]{%
{
\clearpage%
\fontsize{16pt}{14pt}\selectfont
\begin{center}
\textbf{\MakeUppercase{#1}:}
\end{center}
}
}

\newcommand{\anylogic}{\texttt{AnyLogic}}
\newcommand{\labyear}{2017}
\newcommand{\labtitle}{AnyLogic}
\newcommand{\prepod}{Осипов~Н.~А.}
\newcommand{\student}{Кислюк~И.~В.}


% Configurations

% Рис 1. -> Рис 1 --, Таблица 1. -> Таблица 1 --
% Рис 1. -> Рисунок 1
\DeclareCaptionFormat{myformat}{\fontsize{12}{12}\selectfont#1#2#3}
\captionsetup[figure]{format={myformat},name={Рисунок},labelsep=endash}

\begin{document}

	\begin{titlepage}
	\begin{center}	
		\fontsize{14pt}{14pt}\selectfont
		МИНИСТЕРСТВО ОБРАЗОВАНИЯ И НАУКИ\\

		\vspace*{0.6\baselineskip}

		\MakeUppercase{Санкт-Петербургский Национальный Исследовательский Университет Информационных технологий, механики и оптики}		
		
		\vspace*{0.6\baselineskip}
		\MakeUppercase{Факультет Инфокоммуникационных технологий}\\
		\MakeUppercase{Кафедра программных систем}
	
		\vspace*{7\baselineskip}
		\fontseries{m}\fontsize{19pt}{18pt}\selectfont
		Отчет по лабораторной работе
		
		\fontseries{m}\fontsize{20pt}{18pt}\selectfont
		\textbf{\labtitle}\\
		\vspace*{1.15\baselineskip}
		\end{center}
	
	\vspace*{2\baselineskip}
	\begin{flushright}
	\fontseries{m}\fontsize{14pt}{14pt}\selectfont
	\textbf{Выполнил:}\\
	\student\\
	студент группы К4120\\
	Проверил: \prepod\\
	\end{flushright}
	
	\vspace{\fill}
	\begin{center}
	Санкт-Петербург\\
	\vspace{-1ex}
	\labyear
	\end{center}
	
\end{titlepage}

\fontsize{14pt}{14pt}\selectfont

\header{Цель работы}

Изучить и научиться применять пакет моделирования \anylogic.

\header{Теоретическое введение}

\begin{center}
\textbf{Задача Эрланга}
\end{center}

Задача Эрланга связана с разработкой системы массового обслуживания для телефонной сети. Пример задачи -- в систему поступают заявки -– телефонные вызовы с интервалом времени между вызовами, который соответствует экспоненциальному закону распределения с некоторой интенсивностью $a$. Заявки обслуживаются в процессоре. Обслуживание заявки выполняется с некоторой интенсивностью $b$. При обслуживании вызова используется N каналов телефонной сети. Число каналов телефоной станции изменяется. Постройте модель обслуживания телефонных вызовов.

\header{Ход работы}

\begin{enumerate}

\item В данном задании мы построили модель для исследования процесса незатухающих гармонических колебаний, а также удобный пользовательский интерфейс для её использования (рисунок \ref{sshot1}):

\sshot{Модель незатухающих гармонических колебаний}{1}

\clearpage

\item Следующим заданием было создание модели обслуживания сервера, с использованием пользовательского класса в качестве агента. Созданная модель данной системы представлена на рисунке \ref{sshot2}.

\sshot{Модель обслуживания сервера}{2}

\clearpage

\item Далее, мы изучили моделирование конечных автоматов на примере модели светофора (рисунок \ref{sshot3}).

\sshot{Модель светофора}{3}

\clearpage

\item Также мы ознакомились с типовой задачей Эрланга и рассмотрели способы оптимизации моделей в \anylogic {} (рисунки \ref{sshot4}, \ref{sshot5}):

\sshot{Типовая задача Эрланга}{4}
\sshot{Оптимизации модели}{5}

\clearpage

\item В контрольном задании требовалось построить модель функционирования системы связи с двумя каналами -– основным и резервным, с перерывами в работе основного канала. Провести моделирование в течение 5 часов, а также определить рациональную ёмкость накопителя и вероятности передачи сообщений разными каналами (рисунки 6, 7):

\sshot{Оптимизация финальной модели}{6}
\sshot{Модель функционирования системы связи}{7}

\end{enumerate}

\header{Вывод}

Успешно изучили и научились применять пакет \anylogic {} для моделирования систем.


\end{document}

