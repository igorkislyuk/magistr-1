\documentclass[14pt,a4paper]{extreport}

% Caption configuration
\usepackage{caption}

% Times New Roman, be sure to build with XeLaTeX
\usepackage{fontspec}
\usepackage[russian]{babel}
\setmainfont{Times New Roman}

% mock data
\usepackage{lipsum}

% russian
\usepackage[utf8]{inputenc}

\linespread{1.5}

\usepackage{graphicx}
\usepackage[
    left = 2cm,
    right = 1cm,
    top = 1cm,
    bottom = 2cm,
    headheight = 2cm
]{geometry}

%% New commands

% caption | screenshot index
\newcommand{\screenshot}[2]{\begin{figure}[ht]%
\centering\includegraphics[width=0.8\textwidth]{../Screenshots/Screenshot_#2}%
\caption{#1}%
\label{picture#2}%
\end{figure}%
}

\newcommand{\header}[1]{%
{
\clearpage%
\fontsize{16pt}{14pt}\selectfont
\begin{center}
\textbf{\MakeUppercase{#1}:}
\end{center}
}
}

\newcommand{\labyear}{2017}
\newcommand{\labtitle}{Расчет площадей методом Монте-Карло}
\newcommand{\prepod}{Осипов~Н.~А.}
\newcommand{\student}{Кислюк~И.~В.}

% Configurations

% Рис 1. -> Рис 1 --, Таблица 1. -> Таблица 1 --
% Рис 1. -> Рисунок 1
\DeclareCaptionFormat{myformat}{\fontsize{12}{12}\selectfont#1#2#3}
\captionsetup[figure]{format={myformat},name={Рисунок},labelsep=endash}

\begin{document}

	\begin{titlepage}
	\begin{center}	
		\fontsize{14pt}{14pt}\selectfont
		МИНИСТЕРСТВО ОБРАЗОВАНИЯ И НАУКИ\\

		\vspace*{0.6\baselineskip}

		\MakeUppercase{Санкт-Петербургский Национальный Исследовательский Университет Информационных технологий, механики и оптики}		
		
		\vspace*{0.6\baselineskip}
		\MakeUppercase{Факультет Инфокоммуникационных технологий}\\
		\MakeUppercase{Кафедра программных систем}
	
		\vspace*{7\baselineskip}
		\fontseries{m}\fontsize{19pt}{18pt}\selectfont
		Отчет по лабораторной работе
		
		\fontseries{m}\fontsize{20pt}{18pt}\selectfont
		\textbf{\labtitle}\\
		\vspace*{1.15\baselineskip}
		\end{center}
	
	\vspace*{2\baselineskip}
	\begin{flushright}
	\fontseries{m}\fontsize{14pt}{14pt}\selectfont
	\textbf{Выполнил:}\\
	\student\\
	студент группы К4120\\
	Проверил: \prepod\\
	\end{flushright}
	
	\vspace{\fill}
	\begin{center}
	Санкт-Петербург\\
	\vspace{-1ex}
	\labyear
	\end{center}
	
\end{titlepage}

\fontsize{14pt}{14pt}\selectfont

\header{Теоретические сведения}

Метод Монте-Карло (или метод статистических испытаний) можно
определить как \textbf{метод моделирования случайных величин с целью
вычисления характеристик их распределений}. Суть состоит в том, что
результат испытаний зависит от некоторой случайной величины, распределенной по заданному закону. Поэтому результат каждого отдельного испытания носит случайный характер (как правило, составляется программа для осуществления одного случайного испытания). \par
Проведя серию испытаний, получают выборку. Полученные статистические данные обрабатываются и представляются в виде численных оценок интересующих исследователя величин (характеристик системы). \par\textbf{Теоретической основой метода Монте-Карло являются предельные
теоремы теории вероятностей}. Они гарантируют высокое качество статистических оценок при весьма большом числе испытаний. Метод статистических испытаний применим для исследования как стохастических, так и детерминированных систем. \textbf{Практическая реализация метода Монте-Карло невозможна без использования компьютера.}

\header{Цель работы}

Исследовать и научиться применять на практике метод статистических испытаний (Монте-Карло).

\header{Ход работы}

Первым заданием является нахождение квадратуры круга. В качестве исходных данных у нас имеются координаты центра круга и его радиус (т.е. уравнение окружности)

$$
	(x—2)^2  + (у—3)^2 = 25
$$

Впишем данный круг в квадрат 10х10, в котором будем создавать точки, распределённые равномерны. Проведя n испытаний, посчитаем количество m точек, попавших в круг. Тогда оценку площади круга можно будет получить из следующего соотношения:

$$
	S_{кр} = S_{кв} \cdot \frac{m}{n} = 100 \cdot \frac{m}{n}
$$


Построим имитационную модель средствами MatLab (\ref{picture1}):

\screenshot{Имитационная модель для нахождения площади круга}{1}

\par

Однако на основе одного прогона имитационной модели нельзя сделать выводы о результате, поэтому необходимо провести статистический эксперимент (\ref{picture2}).

\screenshot{Данные статистического эксперимента}{2}

Как видно из результатов эксперимента, точность метода Монте-Карло 	увеличивается при значительном увеличении числа испытаний.

\clearpage

Следующим заданием требуется рассчитать определённый интеграл от $a$ до $b$ от функции $y = f(x)$. Это используется для расчёта «неберущихся» интегралов. Для примера воспользуемся следующим уравнением (пример графика показан на рисунке \ref{picture3}

$$
	\int\limits_3^8 -x^2+7x dx
$$

\screenshot{Пример графика интеграла}{3}

Аналогично нахождению площади круга, построим модель и проведём 	статистический эксперимент (рисунки \ref{picture4}, \ref{picture5})

\screenshot{Модель для расчёта определённого интеграла}{4}
\screenshot{Данные второго статистического эксперимента}{5}

\clearpage
\header{Вывод}

Успешно исследовали и применили на практике метод Монте-Карло.

\end{document}

