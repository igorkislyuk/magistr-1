\documentclass[14pt,a4paper]{extreport}

% Caption configuration
\usepackage{caption}

% Times New Roman, be sure to build with XeLaTeX
\usepackage{fontspec}
\usepackage[russian]{babel}
\setmainfont{Times New Roman}

% mock data
\usepackage{lipsum}

% russian
\usepackage[utf8]{inputenc}

\linespread{1.5}

\usepackage{graphicx}
\usepackage[
    left = 2cm,
    right = 1cm,
    top = 1cm,
    bottom = 2cm,
    headheight = 2cm
]{geometry}

%% New commands

% caption | screenshot index
\newcommand{\sshot}[2]{\begin{figure}[ht]%
\centering\includegraphics[width=0.8\textwidth]{../Screenshots/Screenshot_#2}%
\caption{#1}%
\label{sshot#2}%
\end{figure}%
}

\newcommand{\header}[1]{%
{
\clearpage%
\fontsize{16pt}{14pt}\selectfont
\begin{center}
\textbf{\MakeUppercase{#1}:}
\end{center}
}
}

\newcommand{\labyear}{2017}
\newcommand{\labtitle}{Байесовские сети}
\newcommand{\prepod}{Осипов~Н.~А.}
\newcommand{\student}{Кислюк~И.~В.}


% Configurations

% Рис 1. -> Рис 1 --, Таблица 1. -> Таблица 1 --
% Рис 1. -> Рисунок 1
\DeclareCaptionFormat{myformat}{\fontsize{12}{12}\selectfont#1#2#3}
\captionsetup[figure]{format={myformat},name={Рисунок},labelsep=endash}

\begin{document}

	\begin{titlepage}
	\begin{center}	
		\fontsize{14pt}{14pt}\selectfont
		МИНИСТЕРСТВО ОБРАЗОВАНИЯ И НАУКИ\\

		\vspace*{0.6\baselineskip}

		\MakeUppercase{Санкт-Петербургский Национальный Исследовательский Университет Информационных технологий, механики и оптики}		
		
		\vspace*{0.6\baselineskip}
		\MakeUppercase{Факультет Инфокоммуникационных технологий}\\
		\MakeUppercase{Кафедра программных систем}
	
		\vspace*{7\baselineskip}
		\fontseries{m}\fontsize{19pt}{18pt}\selectfont
		Отчет по лабораторной работе
		
		\fontseries{m}\fontsize{20pt}{18pt}\selectfont
		\textbf{\labtitle}\\
		\vspace*{1.15\baselineskip}
		\end{center}
	
	\vspace*{2\baselineskip}
	\begin{flushright}
	\fontseries{m}\fontsize{14pt}{14pt}\selectfont
	\textbf{Выполнил:}\\
	\student\\
	студент группы К4120\\
	Проверил: \prepod\\
	\end{flushright}
	
	\vspace{\fill}
	\begin{center}
	Санкт-Петербург\\
	\vspace{-1ex}
	\labyear
	\end{center}
	
\end{titlepage}

\fontsize{14pt}{14pt}\selectfont

\header{Цель работы}

Создать модель (байесовскую сеть) в среде \texttt{Genie\_{}p} и с её помощью определить необходимые вероятности.

\header{Теоретическое введение}

\begin{center}
\textbf{Теорема Байеса}
\end{center}

Вероятность наступления «события» при условии проведения «наблюдения» равна произведению вероятности наступления события и вероятности проведения наблюдения при условии наступления события, деленному на безусловную вероятность проведения наблюдения.

\par

Байесовская сеть — это направленный ациклический граф, в котором каждая вершина - случайная переменная.

\par

\begin{center}
\textbf{Процесс рассуждения (вывода) в байесовских сетях доверия}
\end{center}

\begin{enumerate}

\item Процесс рассуждения сопровождается распространением по сети вновь поступивших свидетельств. 

\item Введение в байесовские сети доверия новых данных приводит к возникновению переходного процесса распространения по байесовской сети доверия вновь поступившего свидетельства. 

\item После завершения переходного процесса каждому высказыванию, ассоциированному с вершинами графа, приписывается апостериорная вероятность, которая определяет степень доверия к этому высказыванию 

\end{enumerate}

\header{Ход работы}



\begin{enumerate}

\item  В начале работы собираем из элементов необходимую сеть, устанавливаем условные вероятности событий (рисунок \ref{sshot1}):

\sshot{Байесовская сеть}{1}

\clearpage

\item Далее, необходимо определить следующие полные вероятности:

\begin{itemize}

\item диагностике логический диск недоступен
\item на экране монитора нет изображения
\item не горит индикатор питания

\end{itemize}

Они приведены на рисунках \ref{sshot2}--\ref{sshot4}:

\sshot{Вероятность недоступности логического диска}{2}
\sshot{Вероятность отсутствия изображения на экране монитора}{3}
\sshot{Вероятность того, что индикатор питания не горит}{4}

\clearpage

\item Далее, предоставим в сеть следующие свидетельства:

\begin{itemize}

\item на жестком диске появились дефектные секторы

\item контакты блока питания не нарушены

\end{itemize}

Результат на рисунке \ref{sshot5}.

\sshot{Система с двумя свидетельствами}{5}

\end{enumerate}

\header{Вывод}

Успешно создали модель байесовской сети и с её помощью определили вероятности заданных событий.

\end{document}

